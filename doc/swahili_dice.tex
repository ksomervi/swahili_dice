\documentclass[twocolumn,12pt]{article}
\usepackage{parskip}
\usepackage{fancyhdr}
%\documentclass{leaflet}

\textwidth=7in
\oddsidemargin=-0.25in
%\evensidemargin=-.25in
\textheight=9.5in
\topmargin=-1.250in

\begin{document}

\parindent=0in

%\huge  \bf{Swahili Dice}
%\normalsize
\title{\bf{Swahili Dice}}
\author{\bf{Players:} 3 or more}
\date{\bf{Dice:} 5 or 7}
\maketitle

\thispagestyle{empty}
%\pagestyle{fancyplain}
%\rfoot{Fun while drinking.}
%\cfoot{}

\section*{Objective}
To score the most points.

\section*{Start of play}
To start the game each player roles a single dice and the highest score
starts. Play then passes to the left.  To begin scoring, the player must score
500 points or more during a single turn.

\section*{Scoring}
Ones are worth 100 points and fives are worth fifty points.
\subsection*{Rolling three of a kind}
\begin{itemize}
\item Three ones gives the player 1000 points
\item Three of any other value gives the player 100 points times the number
  value of the dice (e.g., three twos is 200 points).
\end{itemize}

Rolling four of a kind doubles the value had from rolling three of a kind.  As
an example, rolling four ones is 2000 points.  Rolling four twos is 400
points.

Rolling five of a kind doubles the value had from rolling four of a kind
except for five ones which wins the game with 5000 points.

\subsection*{Big Roll}
A ``Big Roll'' is when a player rolls all five dice, in a single roll, and the dice
land as one through five. A ``Big Roll'' is worth 500 points.

\section*{Game Play}
During play, each player rolls five dice at the start of their turn.  All point
scoring dice are included (i.e., three or more of a kind, ones, and fives). Any
remaining dice can be rolled again to add to the score.  A turn cannot end
with a 50 point value.  As an example, rolling three twos, a five, and a one
scores 200+50+100 = 350 points.  The player must roll again because the score is
not a multiple of 100 points.

During a roll, if no points are scored, play goes to the next player to the
left.  If
points are scored, the player can roll remaining non-scoring dice to
attempt to earn more points. As
an example, rolling three fours, a two, and a six, the player can roll the
two and six dice.  If no points are scored during a subsequent roll, all points for
the turn are lost (i.e., to keep points, the player has to pass the dice after
a roll earning points). 

In the event that all of the dice have points (from a single or multiple
rolls), the player can either roll all five dice to keep increasing their score
or pass to the next player. If the player rolls and does not score, all points
from the current turn are lost.  If the player passes, the next turn is
forfeit.

\section*{Winning the game}
When a player reaches 5000 points, each
remaining player gets one more turn at rolling. The player with the most
points wins.
\end{document}
